% Abstract
\thispagestyle{plain}
\phantomsection
\addcontentsline{toc}{chapter}{Abstract}

\centerline{\zihao{3}\bfseries ABSTRACT}

\linespread{1.4}\zihao{-4}
\bigskip


Due to the non-local nature of fractional derivatives,
the fractional diffusion equations 
 can be used to describe some abnormal diffusion phenomena in physics.
Therefore, models based on fractional derivatives 
have been increasingly used in the fields of scientific and engineering computing in recent years.
But it is also because of the non-local nature of fractional derivatives,
the coefficient matrix of the discrete linear system is often dense.
When the scale of the problem is large,
it brings great difficulty to compute the numerical solution of the problem.
Fast algorithms for traditional integer-order differential equations
are no longer applicable to fractional-order differential equations.
In recent years,
research on fast algorithms for fractional diffusion equations has received more and more attentions.

The main contributions of this thesis are as follows:
\begin{itemize}

\item[(1)]
We consider the numerical methods for the time-space fractional diffusion equations.
Discretized by the Gr\"unwald difference method in time and the shifted Gr\"unwald difference method in space,
the original problem was finally transformed into a large-scale system of algebraic linear equations.
Its coefficient matrix can be expressed as a sum of two Kronecker products.
\medskip

\item[(2)]
Based on the special structure of the coefficient matrix,
we combine inversion method of Toeplitz matrix and
propose a class of block diagonal preconditioner.
The properties of the preconditioned coefficient matrix are studied.
Numerical experiments show that
when the time fractional derivative is small, such as not exceeding 0.5,
the preconditioner has a good acceleration effect.
\medskip

\item[(3)]
We observe that the coefficient matrix consists of two parts naturally,
and each part has a special structure.
Therefore, we propose a class of matrix splitting preconditioner 
based on the idea of ​​alternating directions.
Some theoretical results are presented. 
In order to reduce the amount of workload, in the real applications, 
we make use of the technique of circulant matrix approximation and
inversion method of Toeplitz matrix.
Numerical tests were carried out to show the performance of the two preconditioners.
\end{itemize}

\bigskip
\noindent\textbf{\zihao{4} Keywords:}
Time-space fractional diffusion equation, Inversion of Toeplitz matrix, Alternating direction iteration, Preconditioning 