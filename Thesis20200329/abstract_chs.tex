%%% 中文摘要
\thispagestyle{plain}
\phantomsection
\addcontentsline{toc}{chapter}{摘\quad 要}

\centerline{\zihao{-3}\heiti 摘\quad 要}

\linespread{1.4}\zihao{-4} \bigskip
由于分数阶导数的非局部性质,
它可以很好地用来描述物理中的一些非正常扩散现象, 
故而基于分数阶导数的模型近些年来被越来越多地应用于科学和工程计算领域.
但也正是因为分数阶导数的非局部性质, 离散后得到的代数方程组的系数矩阵往往是稠密的,
当问题规模较大时, 这给问题的数值求解带来了很大的困难.
传统整数阶微分方程的快速算法不再适用于分数阶微分方程.
近年来, 针对分数阶扩散方程的快速算法的研究受到了越来越多的关注. 

本文的主要研究内容如下:
\begin{itemize}

\item[(1)] 
我们考虑了时间-空间分数阶扩散方程的预处理快速算法,
通过时间上的 Gr\"unwald 差分方法离散和空间上的带平移 Gr\"unwald 差分方法离散,
并经过整理, 原问题最终转化为一个大规模的代数方程组, 
其系数矩阵可表示为两个Kronecker乘积之和.
\medskip

\item[(2)]
基于系数矩阵的特殊结构, 我们结合 Toeplitz 
矩阵求逆方法构造了一类块对角预处理子, 并进行了理论分析.
数值实验表明, 当时间分数阶导数较小时, 比如不超过0.5时, 该预处理子具有很好的加速效果.
\medskip

\item[(3)]
观察到系数矩阵由两部分组成, 而且每一部分都具有特殊的结构,
因此我们提出了一类基于交替方向思想的矩阵分裂预处理子, 并对其进行了理论分析.
在具体实施时, 为了减少运算量, 我们分别考虑了用循环矩阵近似和
Toeplitz 矩阵求逆两种方法, 并进行了数值测试.

\end{itemize}

\bigskip

\noindent{\zihao{4}\heiti 关键词:}
% 关键词, 关键词, 关键词
时间-空间分数阶扩散方程, Toeplitz 矩阵求逆, 交替方向迭代, 预处理